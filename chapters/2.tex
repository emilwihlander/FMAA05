%Chapter 2
\chapter{2}{Algebra}
\subsection*{Räkneoperationer för reella tal}

\begin{task}{2.1 a)}
	Två alternativa lösningsmetoder:
	\[(x+3)(x-3)-(x+3)^2=\cancel{x^2}-9-(\cancel{x^2}+6x+9)=-6x-18\]
	eller
	\[(x+3)(x-3)-(x+3)^2=(x+3)((\cancel{x}-3)-(\cancel{x}+3))=-6(x+3)=-6x-18\]
	
	\ans $-6x-18$
\end{task}

\begin{task}{b)}
	Två alternativa lösningsmetoder:
	\[(x+3)(x-3)-(x-3)^2=\cancel{x^2}-9-(\cancel{x^2}-6x+9)=6x-18\]
	eller
	\[(x+3)(x-3)-(x-3)^2=(x-3)((\cancel{x}+3)-(\cancel{x}-3))=6(x-3)=6x-18\]
	
	\ans $6x-18$
\end{task}

\begin{task}{c)}
	$(3x+5)^2-(3x-5)^2=\cancel{9x^2}+30x+\cancel{25}-(\cancel{9x^2}-30x+\cancel{25})=60x$
	
	\ans $60x$
\end{task}

\begin{task}{2.2}
	$(a-b)^3=a^3-3a^2b+3ab^2-b^3$ 
	
	\ans Varannan term är positiv och varannan negativ och antalet av varje term följer Pascals triangel.
\end{task}