%Chapter 2
\chapter{2}{Algebra}
\subsection*{Räkneoperationer för reella tal}

\begin{task}{2.1 a)}
	Två alternativa lösningsmetoder:
	\[(x+3)(x-3)-(x+3)^2=\cancel{x^2}-9-(\cancel{x^2}+6x+9)=-6x-18\]
	eller
	\[(x+3)(x-3)-(x+3)^2=(x+3)((\cancel{x}-3)-(\cancel{x}+3))=-6(x+3)=-6x-18\]
	
	\ans $-6x-18$
\end{task}

\begin{task}{b)}
	Två alternativa lösningsmetoder:
	\[(x+3)(x-3)-(x-3)^2=\cancel{x^2}-9-(\cancel{x^2}-6x+9)=6x-18\]
	eller
	\[(x+3)(x-3)-(x-3)^2=(x-3)((\cancel{x}+3)-(\cancel{x}-3))=6(x-3)=6x-18\]
	
	\ans $6x-18$
\end{task}

\begin{task}{c)}
	$(3x+5)^2-(3x-5)^2=\cancel{9x^2}+30x+\cancel{25}-(\cancel{9x^2}-30x+\cancel{25})=60x$
	
	\ans $60x$
\end{task}

\begin{task}{2.2}
	$(a-b)^3=a^3-3a^2b+3ab^2-b^3$ 
	
	\ans Varannan term är positiv och varannan negativ och antalet av varje term följer Pascals triangel.
\end{task}

\begin{task}{2.3}
	Se konjugatregeln samt tipset till uppgiften.
	\begin{align*}
	(a+b)(a^2+b^2)(a^4+b^4)(a^8+b^8)(a^{16}+b^{16})&=\frac{a^{32}-b^{32}}{a-b} \\
	(a^2-b^2)(a^2+b^2)(a^4+b^4)(a^8+b^8)(a^{16}+b^{16})&=a^{32}-b^{32} \\
	(a^4-b^4)(a^4+b^4)(a^8+b^8)(a^{16}+b^{16})&=a^{32}-b^{32} \\
	(a^8-b^8)(a^8+b^8)(a^{16}+b^{16})&=a^{32}-b^{32} \\
	(a^{16}-b^{16})(a^{16}+b^{16})&=a^{32}-b^{32} \\
	a^{32}-b^{32}&=a^{32}-b^{32} \\
	a^{32}-b^{32}&=a^{32}-b^{32}\mproof
	\end{align*}
\end{task}

\pagebreak
\begin{task}{2.4}
	faktorisera och förenkla:
	\[\dfrac{2}{7}\]
	\[\dfrac{4}{9}=\dfrac{2*2}{3*3}=\dfrac{4}{9}\]
	\[\dfrac{4}{14}=\dfrac{\cancel{2}*2}{\cancel{2}*7}=\dfrac{2}{7}\]
	\[\dfrac{48}{168}=\dfrac{2*\cancel{2*2*2*3}}{\cancel{2*2*2*3}*7}=\dfrac{2}{7}\]
	\[\dfrac{24}{84}=\dfrac{2*\cancel{2*2*3}}{\cancel{2*2*3}*7}=\dfrac{2}{7}\]
	multiplicera med 1000000 (flytta decimaltecknet 6 steg):
	\[\dfrac{0.00002}{0.000007}=\dfrac{20}{7}\]
	
	\ans $\frac{2}{7},~~ \frac{4}{14},~~ \frac{48}{168},~~ \frac{24}{84}$
\end{task}