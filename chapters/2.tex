%Chapter 2
\chapter{2}{Algebra}
\subsection*{Räkneoperationer för reella tal}

\begin{task}{2.1 a)}
	Två alternativa lösningsmetoder:
	\[(x+3)(x-3)-(x+3)^2=\cancel{x^2}-9-(\cancel{x^2}+6x+9)=-6x-18\]
	eller
	\[(x+3)(x-3)-(x+3)^2=(x+3)((\cancel{x}-3)-(\cancel{x}+3))=-6(x+3)=-6x-18\]
	
	\ans $-6x-18$
\end{task}

\begin{task}{b)}
	Två alternativa lösningsmetoder:
	\[(x+3)(x-3)-(x-3)^2=\cancel{x^2}-9-(\cancel{x^2}-6x+9)=6x-18\]
	eller
	\[(x+3)(x-3)-(x-3)^2=(x-3)((\cancel{x}+3)-(\cancel{x}-3))=6(x-3)=6x-18\]
	
	\ans $6x-18$
\end{task}

\begin{task}{c)}
	$(3x+5)^2-(3x-5)^2=\cancel{9x^2}+30x+\cancel{25}-(\cancel{9x^2}-30x+\cancel{25})=60x$
	
	\ans $60x$
\end{task}

\begin{task}{2.2}
	$(a-b)^3=a^3-3a^2b+3ab^2-b^3$ 
	
	\ans Varannan term är positiv och varannan negativ och antalet av varje term följer Pascals triangel.
\end{task}

\begin{task}{2.3}
	Se konjugatregeln samt tipset till uppgiften.
	\begin{align*}
	(a+b)(a^2+b^2)(a^4+b^4)(a^8+b^8)(a^{16}+b^{16})&=\frac{a^{32}-b^{32}}{a-b} \\
	(a^2-b^2)(a^2+b^2)(a^4+b^4)(a^8+b^8)(a^{16}+b^{16})&=a^{32}-b^{32} \\
	(a^4-b^4)(a^4+b^4)(a^8+b^8)(a^{16}+b^{16})&=a^{32}-b^{32} \\
	(a^8-b^8)(a^8+b^8)(a^{16}+b^{16})&=a^{32}-b^{32} \\
	(a^{16}-b^{16})(a^{16}+b^{16})&=a^{32}-b^{32} \\
	a^{32}-b^{32}&=a^{32}-b^{32} \\
	a^{32}-b^{32}&=a^{32}-b^{32}\mproof
	\end{align*}
\end{task}

\pagebreak
\begin{task}{2.4}
	faktorisera och förenkla:
	\[\dfrac{2}{7}\]
	\[\dfrac{4}{9}=\dfrac{2*2}{3*3}=\dfrac{4}{9}\]
	\[\dfrac{4}{14}=\dfrac{\cancel{2}*2}{\cancel{2}*7}=\dfrac{2}{7}\]
	\[\dfrac{48}{168}=\dfrac{2*\cancel{2*2*2*3}}{\cancel{2*2*2*3}*7}=\dfrac{2}{7}\]
	\[\dfrac{24}{84}=\dfrac{2*\cancel{2*2*3}}{\cancel{2*2*3}*7}=\dfrac{2}{7}\]
	multiplicera med 1000000 (flytta decimaltecknet 6 steg):
	\[\dfrac{0.00002}{0.000007}=\dfrac{20}{7}\]
	
	\ans $\frac{2}{7},~~ \frac{4}{14},~~ \frac{48}{168},~~ \frac{24}{84}$
\end{task}

\begin{task}{2.5 a)}
	\[\frac{1}{7}-\left(\frac{15}{14}+\frac{1}{2}\right)=\frac{2}{14}-\left(\frac{15}{14}+\frac{7}{14}\right)=\frac{2}{14}-\frac{22}{14}=-\frac{20}{14}=-\frac{10}{7}\]
	
	\ans $-\frac{10}{7}$
\end{task}

\begin{task}{b)}
	\[\frac{5}{6}-\left(\frac{3}{4}+\frac{1}{3}\right)=\frac{10}{12}-\left(\frac{9}{12}+\frac{4}{12}\right)=\frac{10}{12}-\frac{13}{12}=-\frac{3}{12}=-\frac{1}{4}\]
	
	\ans $-\frac{1}{4}$
\end{task}

\begin{task}{2.6 a)}
	Faktorisera, hitta minsta gemensamma nämnare och förläng.
	\begin{align*}
	\frac{1}{60}+\frac{1}{108}-\frac{1}{72}=&
	\frac{1}{5*3*2*2}+\frac{1}{9*3*2*2}-\frac{1}{6*3*2*2}= \\
	=&\frac{1*9*6}{5*12*9*6}+\frac{1*5*6}{9*12*5*6}-\frac{1*9*5}{6*12*9*5} \\
	=&\frac{54}{3240}+\frac{30}{3240}-\frac{45}{3240}=\frac{39}{3240}=\frac{13}{1080}
	\end{align*}
	
	\ans $\frac{13}{1080}$
\end{task}

\begin{task}{b)}
	Faktorisera, hitta minsta gemensamma nämnare och förläng.
	\[\frac{3}{4}-\frac{5}{6}+\frac{1}{9}=\frac{27}{36}-\frac{30}{36}+\frac{4}{36}=\frac{1}{36}\]
	
	\ans $\frac{1}{36}$
\end{task}

\begin{task}{c)}
	Faktorisera, hitta minsta gemensamma nämnare och förläng stegvis.
	\[\frac{1}{35}-\frac{1}{25}+\frac{1}{63}-\frac{1}{245}=
	\frac{6}{245}-\frac{1}{25}+\frac{1}{63}=
	\frac{89}{2205}-\frac{1}{25}=
	\frac{445}{11025}-\frac{441}{11025}=
	\frac{4}{11025}\]
	
	\ans $\frac{4}{11025}$
\end{task}

\begin{task}{2.7 a)}
	Utnyttja reglerna för division.
	\[\dfrac{\frac{a}{2}}{\frac{a}{4}}=
	\dfrac{\cancel{a}*4}{2*\cancel{a}}=
	\dfrac{4}{2}=
	2\]
	
	\ans 2
\end{task}

\begin{task}{b)}
	Utnyttja reglerna för division.
	\[\dfrac{\frac{a}{2}}{\frac{4}{a}}=
	\dfrac{a*a}{2*4}=
	\dfrac{a^2}{8}\]
	
	\ans $\dfrac{a^2}{8}$
\end{task}

\begin{task}{c)}
	Utnyttja reglerna för division och faktorisera.
	\[\dfrac{\frac{14a}{a+2}}{\frac{7}{6a+12}}=
	\dfrac{\cancelto{2}{14}a(6a+12)}{\cancel{7}(a+2)}=
	\dfrac{2a^2+24a}{a+2}=\dfrac{12a\cancel{(a+2)}}{\cancel{a+2}}=
	12a\]
	
	\ans $12a$
\end{task}

\begin{task}{d)}
	Utnyttja reglerna för division och faktorisera.
	\[\dfrac{\frac{a}{a+3}}{a^2+3a}=
	\dfrac{a}{(a+3)(a^2+3a)}=
	\dfrac{\cancel{a}}{\cancel{a}(a+3)(a+3)}=
	\dfrac{1}{(a+3)^2}=
	(a+3)^{-2}\]
	
	\ans $(a+3)^{-2}$ eller $\dfrac{1}{(a+3)^2}$
\end{task}

\begin{task}{2.8 a)}
	Skriv först ihop de övre och undre bråken. Utnyttja sedan reglerna för division och faktorisera.
	\begin{align*}
		&\dfrac{\frac{3}{5x}-\frac{x}{15}}{\frac{1}{x}-\frac{1}{3}}=
		\dfrac{\frac{45-5x^2}{75x}}{\frac{3-x}{3x}}=
		\dfrac{\cancel{3x}(45-5x^2)}{\cancelto{25}{75}\cancel{x}(3-x)}=
		\dfrac{45-5x^2}{75-25x}= \\
		&=\dfrac{\cancel{5}(9-x^2)}{\cancel{5}(15-5x)}=
		\dfrac{9-x^2}{15-5x}=
		\dfrac{(3+x)\cancel{(3-x)}}{5\cancel{(3-x)}}=
		\dfrac{3+x}{5}
	\end{align*}
	\ans $\dfrac{3+x}{5}$
\end{task}

\begin{task}{b)}
	Skriv först ihop $1+\frac{1}{x^2}$. Utnyttja sedan reglerna för division.
	\[\dfrac{x^2+1}{1+\frac{1}{x^2}}=
	\dfrac{x^2+1}{\frac{x^2+1}{x^2}}=
	\dfrac{\cancel{(x^2+1)}x^2}{\cancel{(x^2+1)}}=
	x^2\]
	\ans $x^2$
\end{task}

\begin{task}{c)}
	Skriv först ihop de övre bråken. Utnyttja sedan reglerna för division och faktorisera ut $-1$.
	\[\dfrac{\frac{1}{x}-\frac{1}{y}}{\frac{x^2-y^2}{(xy)^2}}=
	\dfrac{\frac{y-x}{xy}}{\frac{x^2-y^2}{(xy)^2}}=
	\dfrac{(y-x)(xy)^{\cancel{2}}}{\cancel{xy}(x^2-y^2)}=
	\dfrac{(y-x)(xy)}{(x-y)(x+y)}=
	\dfrac{\cancel{(y-x)}(xy)}{-\cancel{(y-x)}(x+y)}=
	-\dfrac{xy}{x+y}\]
	\ans $-\dfrac{xy}{x+y}$
\end{task}

\begin{task}{2.9 a)}
	Skriv först ihop de övre och undre bråken. Utnyttja sedan reglerna för division och den andra kvadreringsregeln bakvänt.
	\[\dfrac{\frac{x}{y}- \frac{y}{x}}{\frac{x}{y}+\frac{y}{x}-2}=
	\dfrac{\frac{x^2-y^2}{yx}}{\frac{x^2+y^2-2xy}{yx}}=
	\dfrac{\cancel{yx}(x+y)\cancel{(x-y)}}{\cancel{yx}(x-y)^{\cancel{2}}}=
	\dfrac{x+y}{x-y}\]
	\ans $\dfrac{x+y}{x-y}$
\end{task}

\begin{task}{b)}
	Skriv först ihop de övre och undre bråken. Utnyttja sedan reglerna för division och konjugatregeln bakvänt två gånger.
	\begin{align*}
	&\dfrac{\frac{16x^4}{81}-y^4}{\frac{2x}{3}+y}=
	\dfrac{\frac{16x^4-81y^4}{81}}{\frac{2x+3y}{3}}=
	\dfrac{\cancel{3}(16x^4-81y^4)}{\cancelto{27}{81}(2x+3y)}= 
	\dfrac{(4x^2+9y^2)(4x^2-9y^2)}{27(2x+3y)}=\\
	&=\dfrac{(4x^2+9y^2)\cancel{(2x+3y)}(2x-3y)}{27\cancel{(2x+3y)}}=
	\dfrac{8x^3-12x^2y+18xy^2-27y^3}{27}= \\
	&=\dfrac{1}{27}(8x^3-12x^2y+18xy^2-27y^3)
	\end{align*}
	eller
	\begin{align*}
	&\ldots\dfrac{(4x^2+9y^2)\cancel{(2x+3y)}(2x-3y)}{27\cancel{(2x+3y)}}=
	\dfrac{(4x^2+9y^2)(2x-3y)}{9*3}= \\
	&=\dfrac{4x^2+9y^2}{9}*\dfrac{2x-3y}{3}=
	\left(\dfrac{4x^2}{9}+y^2\right)\left(\dfrac{2x}{3}-y\right)
	\end{align*}
	\ans $\dfrac{1}{27}(8x^3-12x^2y+18xy^2-27y^3)$ eller $\left(\dfrac{4x^2}{9}+y^2\right)\left(\dfrac{2x}{3}-y\right)$
\end{task}

\begin{task}{c)}
	Skriv ihop de övre och undre bråken.
	\[\dfrac{\frac{1}{x+1}+\frac{1}{x-1}}{\frac{1}{x-1}-\frac{1}{x+1}}=
	\dfrac{\frac{x-1+x+1}{(x+1)(x-1)}}{\frac{x+1-(x-1)}{(x+1)(x-1)}}=
	\dfrac{(x-\cancel{1}+x+\cancel{1})\cancel{(x+1)(x-1)}}{(\cancel{x}+1-(\cancel{x}-1))\cancel{(x+1)(x-1)}}=
	\dfrac{\cancel{2}x}{\cancel{2}}=
	x\]
	\ans $x$
\end{task}

\begin{task}{2.10 a)}
	Sätt in i formeln och förläng till minsta gemensamma nämnare.
	\[\frac{1}{R}=\frac{1}{2}+\frac{1}{3}+\frac{1}{4} \lra
	\frac{1}{R}=\frac{6}{12}+ \frac{4}{12}+\frac{3}{12} \lra
	\frac{1}{R}=\frac{13}{12} \lra
	R=\frac{12}{13}\Omega\]
	\ans $\dfrac{12}{13}\Omega$
\end{task}

\begin{task}{b)}
	Använd räkneregler för division.
	\[\frac{1}{3}=\frac{1}{5}+\frac{1}{R} \lra
	\frac{1}{3}=\frac{R+5}{5R} \lra
	5x=3R+15 \lra
	2R=15 \lra
	R=\frac{15}{2} \Omega\]
	\ans $\dfrac{12}{13}\Omega$
\end{task}

\begin{task}{2.11}
	Sätt in i formel och använd räkneregler för division.
	\begin{align*}
	& \frac{1}{a}+\frac{1}{600}=\frac{1}{100} \lra
	\frac{600+a}{600a}=\frac{1}{100} \lra
	60000+100a=600a \lra \\ \lra
	& 500a=60000 \lra
	a=\frac{60000}{500}=120\text{ mm}
	\end{align*}
	\ans $120$ mm
\end{task}

\begin{task}{2.12}
	Utnyttja att $1=2/2=3/3$ osv. och använd räkneregler för division.
	\[\dfrac{1}{1+\frac{1}{1+\frac{1}{1+1}}}=
	\dfrac{1}{1+\frac{1}{1+\frac{1}{2}}}=
	\dfrac{1}{1+\frac{1}{\frac{2+1}{2}}}=
	\dfrac{1}{1+\frac{2}{3}}=
	\dfrac{1}{\frac{5}{3}}=
	\dfrac{3}{5}\]
	\ans $\dfrac{3}{5}$
\end{task}

\begin{task}{2.13}
	Förläng med konjugatet för att bli av med roten i nämnaren.
	\[\dfrac{3+\sqrt{5}}{2+\sqrt{5}}=
	\dfrac{(3+\sqrt{5})(2-\sqrt{5})}{(2+\sqrt{5})(2-\sqrt{5})}=
	\dfrac{6-3\sqrt{5}+2\sqrt{5}-5}{4-5}=
	-(6-\sqrt{5}-5)=
	\sqrt{5}-1\]
	\ans $\sqrt{5}-1$
\end{task}

\begin{task}{2.14 a)}
	Förläng med konjugatet.
	\[\dfrac{1+2\sqrt{2}}{3-\sqrt{2}}=
	\dfrac{(1+2\sqrt{2})(3+\sqrt{2})}{(3-\sqrt{2})(3+\sqrt{2})}=
	\dfrac{3+\sqrt{2}+6\sqrt{2}+4}{9-2}=
	\dfrac{\cancel{7}+\cancel{7}\sqrt{2}}{\cancel{7}}=
	1+\sqrt{2}\]
	\ans $1+\sqrt{2}$
\end{task}

\begin{task}{b)}
	Förläng med konjugatet.
	\[\dfrac{1}{\sqrt{13}+\sqrt{11}}=
	\dfrac{\sqrt{13}-\sqrt{11}}{(\sqrt{13}+\sqrt{11})(\sqrt{13}-\sqrt{11})}=
	\dfrac{\sqrt{13}-\sqrt{11}}{13-11}=
	\dfrac{\sqrt{13}-\sqrt{11}}{2}\]
	\ans $\dfrac{\sqrt{13}-\sqrt{11}}{2}$
\end{task}

\begin{task}{c)}
	Förläng med konjugatet.
	\begin{align*}
	&\dfrac{2}{\sqrt{x+1}+\sqrt{x-1}}=
	\dfrac{2(\sqrt{x+1}-\sqrt{x-1})}{(\sqrt{x+1}+\sqrt{x-1})(\sqrt{x+1}-\sqrt{x-1})}= \\ =
	&\dfrac{2(\sqrt{x+1}-\sqrt{x-1})}{\cancel{x}+1-(\cancel{x}-1)}=
	\dfrac{\cancel{2}(\sqrt{x+1}-\sqrt{x-1})}{\cancel{2}}=
	\sqrt{x+1}-\sqrt{x-1}
	\end{align*}
	\ans $\sqrt{x+1}-\sqrt{x-1}$
\end{task}

\begin{task}{2.15 a)}
	Faktorisera ena roten för att få samma rot i båda termerna.
	\[\sqrt{12}-\sqrt{3}=
	2\sqrt{3}-\sqrt{3}=
	\sqrt{3}(2-1)=
	\sqrt{3}\]
	\ans $\sqrt{3}$
\end{task}

\begin{task}{b)}
	faktorisera täljaren.
	\[\dfrac{\sqrt{42}}{\sqrt{6}}=
	\dfrac{\cancel{\sqrt{6}}\sqrt{7}}{\cancel{\sqrt{6}}}=
	\sqrt{7}\]
	eller utnyttja reglerna för division med rötter.
	\[\dfrac{\sqrt{42}}{\sqrt{6}}=\sqrt{\dfrac{42}{6}}=\sqrt{7}\]
	\ans $\sqrt{7}$
\end{task}

\begin{task}{c)}
	Faktorisera.
	\[\sqrt{3}*\sqrt{12}=
	\sqrt{3}*\sqrt{3}*\sqrt{4}=
	3*2=6\]
	eller utnyttja reglerna för multiplikation med rötter.
	\[\sqrt{3}*\sqrt{12}=
	\sqrt{3*12}=
	\sqrt{36}=
	6\]
	\ans $6$
\end{task}

\begin{task}{d)}
	Faktorisera termerna i täljaren och skriv ihop.
	\[\dfrac{\sqrt{18}+\sqrt{8}}{5}=
	\dfrac{3\sqrt{2}+2\sqrt{2}}{5}=
	\dfrac{\cancel{5}\sqrt{2}}{\cancel{5}}=
	\sqrt{2}\]
	\ans $\sqrt{2}$
\end{task}

\begin{task}{e)}
	Addera termerna i roten.
	\[\sqrt{3^2+4^2}-4-3=
	\sqrt{25}-7=
	5-7=
	-2\]
	\ans $-2$
\end{task}

\begin{task}{f)}
	Addera termerna i roten.
	\[\sqrt{5^2+12^2}=\sqrt{169}=13\]
	\ans $13$
\end{task}

\begin{task}{2.16 a)}
	Faktorisera rötterna så alla termerna får $\sqrt{2}$ gemensamt.
	\[\dfrac{\sqrt{168}+\sqrt{98}}{\sqrt{50}+\sqrt{2}}=
	\dfrac{9\sqrt{2}+7\sqrt{2}}{5\sqrt{2}+\sqrt{2}}=
	\dfrac{\cancel{\sqrt{2}}(9+7)}{\cancel{\sqrt{2}}(5+1)}=
	\dfrac{9+7}{5+1}=
	\dfrac{16}{6}=
	\dfrac{8}{3}\]
	\ans $\dfrac{8}{3}$
\end{task}

\begin{task}{b)}
	Kvadrera under ``roten ur''-tecknet först (minustecknet försvinner). 
	\[\dfrac{\sqrt{(-4)^2}}{\sqrt{4^2}}=
	\dfrac{4}{4}=
	1\]
	\ans $1$
\end{task}

\begin{task}{c)}
	Skriv först ihop termerna utnyttja sedan reglerna för multiplikation av rötter.
	\[\left(\sqrt{12}-\frac{1}{\sqrt{3}}\right)^2=
	\left(\frac{\sqrt{12}\sqrt{3}-1}{\sqrt{3}}\right)^2=
	\left(\frac{\sqrt{36}-1}{\sqrt{3}}\right)^2=
	\left(\frac{5}{\sqrt{3}}\right)^2=
	\frac{25}{3}\]
	Eller så används den andra kvadreringsregeln.
	\[\left(\sqrt{12}-\frac{1}{\sqrt{3}}\right)^2=
	12-2*\dfrac{\sqrt{12}}{\sqrt{3}}+\dfrac{1}{3}=
	12-2*\sqrt{\dfrac{12}{3}}+\dfrac{1}{3}=
	12-4+\dfrac{1}{3}=
	8+\dfrac{1}{3}=
	\dfrac{24+1}{3}=
	\dfrac{25}{3}\]
	\ans $\dfrac{25}{3}$
\end{task}

\begin{task}{d)}
	Använd konjugatregeln och sen kvadreringsregeln.
	\[((\sqrt{x}+\sqrt{y})+\sqrt{x+y})((\sqrt{x}+\sqrt{y})-\sqrt{x+y})=
	(\sqrt{x}+\sqrt{y})^2-(x+y)=
	x+2\sqrt{xy}+y-x-y=
	2\sqrt{xy}\]
	\ans $2\sqrt{xy}$
\end{task}

\begin{task}{2.17}
	\[\dfrac{\sqrt{216}}{3\sqrt{2}}=
	\dfrac{\sqrt{9*2*4*3}}{3\sqrt{2}}=
	\dfrac{\cancel{3\sqrt{2}}*2\sqrt{3}}{\cancel{3\sqrt{2}}}=
	2\sqrt{3}\]
	\[\dfrac{\sqrt{108}}{3}=
	\dfrac{\sqrt{9*4*3}}{3}=
	\dfrac{\cancel{3}*2\sqrt{3}}{\cancel{3}}=
	2\sqrt{3}\]
	\[\sqrt{12}=\sqrt{4*3}=2\sqrt{3}\]
	\ans Möjligt att någon får poängavdrag på grund av att hen inte förenklat, alla är dock samma tal.
\end{task}

\begin{task}{2.18 a)}
	Använd räkneregler för potenser.
	\[3^4*3^2=3^{(4+2)}=3^6\]
	\ans $3^6$
\end{task}

\begin{task}{b)}
	Använd räkneregler för potenser.
	\[2^7*2^{-3}=2^{7-3}=2^4\]
	\ans $2^4$
\end{task}

\begin{task}{c)}
	Använd räkneregler för potenser.
	\[4^2*4^{-5}*4=4^{2-5+1}=4^{-2}\]
	\ans $4^{-2}$
\end{task}

\begin{task}{d)}
	Använd räkneregler för potenser.
	\[\dfrac{3^7}{3^3}=3^{7-3}=3^4\]
	\ans $3^4$
\end{task}