%Chapter 1
\chapter{1}{Grundläggande begrepp och terminologi}
\subsection*{Talsystem}

\begin{task}{1.1 a)}
	De naturliga talen ($\mathbb{N}$) innefattar alla heltal som är noll eller större. $\tfrac{6}{2}=3,~\tfrac{3}{0.1}=30,~\tfrac{0}{5}=0$.
	
	\ans $\frac{6}{2},~0,~3,~\frac{3}{0.1},~\frac{0}{5}$
\end{task}

\begin{task}{b)}
	De hela talen ($\mathbb{Z}$) inkluderar de naturliga talen ($\mathbb{N}$) samt alla negativa heltal. $-\frac{0.3}{0.02}=-15$.

	\ans $\frac{6}{2},~0,~3,~-3,~\frac{3}{0.1},~-\frac{0.3}{0.02},~\frac{0}{5}$
\end{task}

\begin{task}{c)}
	Rationella tal ($\mathbb{Q}$) är tal som kan skrivas som bråk (inkluderar de hela talen ($\mathbb{Z}$)). $3=\frac{3}{1}$ osv\ldots
	
	\ans $\frac{6}{2},~0,~3,~-3,~\frac{3}{0.1},~\frac{3}{5},~\frac{5}{3}~-\frac{0.3}{0.02},~\frac{0}{5}$
\end{task}

\begin{task}{d)}
	Reella tal ($\mathbb{R}$) är alla ``vanliga'' tal (inte de komplexa talen ($\mathbb{C}$)).
	
	\ans $\frac{6}{2},~0,~3,~-3,~\frac{3}{0.1},~\frac{3}{5},~\frac{5}{3},~\sqrt{2},~-\frac{0.3}{0.02},~\frac{0}{5},~\pi$
\end{task}

\begin{task}{1.2}
	Alla tal med ändligt antal decimaler kan skrivas som rationella tal ($1.41421=\frac{141421}{100000}$). Vi antar att ett irrationellt tal $i$ plus ett rationellt tal $r_1$ blir det rationella talet $r_2$. $i+r_1=r_2 \lra i=r_2-r_1$. Eftersom alla bråk går att skriva ihop som ett bråk stämmer inte antagandet. Svaret måste alltså bli irrationellt.
	
	\ans Nej, båda blir irrationella.
\end{task}

\subsection*{Mängder och intervall}

\begin{task}{1.3}
	$M_1=\{-1,1\}$, eftersom $(-1)^2=1$ och $1^2=1$.
	
	$M_2$ är alla tal större än eller lika med 0.
	
	$M_3$ är alla tal större än eller lika med 1.
	
	$M_4=\mathbb{R}$, eftersom alla reella tal upphöjt i 2 är positivt.
	
	Eftersom $M_4$ är alla tal ingår $M_1,~M_2,~M_3$ i mängden. $M_3$ är även en delmängd av $M_2$.
	
	\ans $M_1 \subseteq M_4,~ M_3 \subseteq M_2 \subseteq M_4$
\end{task}

\subsection*{Implikationer och ekvivalens}

\begin{task}{1.4}
	Eftersom $x^2<16 = -4<x<4$ så betyder det att $A$ och $C$ är ekvivalenta och
	eftersom x alltid är större än $-4$ i $C$ implicerar, både $A$ och $C$, $B$.
	
	\ans $A \Leftrightarrow C,~ C \ra B,~ A \ra B$
\end{task}

\begin{task}{1.5 a)}
	Om $A$ är sant är $B$ sant men om $B$ är sant behöver inte $A$ vara sant. Detta eftersom $a=1,~b=-1$ är sant för $B$ men inte för $A$. $A$ implicerar alltså $B$. $C$ går att förenkla till $a=b$ genom att dela på $b$ det medför dock att $b\neq0$. Eftersom en lösning är att $b=0,~a\in\mathbb{R}$ så är de inte ekvivalenta utan $A$ implicerar $C$. $C$ och $B$ är skilda från varandra eftersom inget av de två ovan nämnda fallen passar in på båda utsagorna. 
	
	\ans $A \ra B,~ A \ra C$
\end{task}

\begin{task}{b)}
	Eftersom specialfallen som nämns i \taskref{a)} båda kräver tal som är mindre än eller lika med 0 (och att det inte finns andra specialfall) är $A$, $B$ och $C$ ekvivalenta. Om man kvadrerar båda sidorna i $D$ får man $A$ vilket medför att även $D$ är ekvivalent med alla andra utsagor.
	
	\ans Alla utsagor är ekvivalenta.
\end{task}

\begin{task}{1.6}
	$A$ ger sant för alla tal större än noll. $B$ ger sant för alla tal utom noll. $C$ ger sant för alla tal utom noll. $D$ ger sant för alla tal större än noll. 
	
	$A$ och $D$ är alltså ekvivalenta, lika så $B$ och $C$. $A\subseteq B$ medför då att $A$ och $D$ implicerar både $B$ och $C$.
	
	\ans $A \ra B,~ A \ra C,~ D \ra B,~ D \ra C,~ A \Leftrightarrow D,~ B \Leftrightarrow C$
\end{task}

\begin{task}{1.7}
	\[A:~x^2-3x+2=0 \rightarrow x = \frac{3}{2} \pm \sqrt{\frac{1}{4}} \rightarrow x_1 = 2,~x_2 = 1\]
	\[B:~|x-2|=1 \rightarrow x=\pm 1+2 \rightarrow x_1=1,~x_2=3\]
	\[C:~x \ge 1\]
	\[D:~lnx + ln(x^3) = 0 \rightarrow x=1\]
	$D$ ingår i alla andra vilket medför att $D$ implicerar alla andra. Eftersom svaren i både $A$ och $B$ är större än eller lika med 1 implicerar $A$ och $B$ $C$.
	
	\ans $D \ra A,~ D \ra B,~ D \ra C,~ A \ra C,~ B \ra C$
\end{task}

\begin{task}{1.8}
	\[A:~x\ge 0\]
	\[B:~\ln x \ge 0 \lra x\ge 1\]
	\[C:~e^x \ge 0 \lra x \in \mathbb{R}\]
	\[D:~|x-2|<1 \lra x-2<1,~x-2>-1 \lra 1<x<3\]
	Alla implicerar $C$ eftersom $C$ är alla tal. $D$ är en delmängd av $B$ som i sin tur är en delmängd av $A$. $D$ implicerar alltså $A$ och $B$ och $B$ implicerar $A$.
	
	\ans $A \ra C,~ D \ra A,~ B \ra A,~ D \ra B,~ D \ra C,~ B \ra C$
\end{task}

\begin{task}{1.9}
	\[A:~|x|>0 \lra x\neq 0\]
	\[B:~e^x > 1 \lra x > 0\]
	\[C:~\cos x \le 1 \lra x \in \mathbb{R}\]
	\[D:~\ln(1+x^2)>0 \lra 1+x^>e^0 \Leftrightarrow x^2>1-1 \lra x\neq 0\]
	$B\ra A$ är alltså sant ($x > 0\subseteq x\neq 0$), $A$ och $B$ är alltså inte samma mängd. $C$ implicerar inte $D$ eftersom $C$ innehåller 0 vilket $D$ inte gör. $A$ och $D$ är däremot ekvivalenta och implicerar $C$.
	
	\ans $B \ra A,~ A \ra C,~ A \Leftrightarrow D$
\end{task}

\begin{task}{1.10}
	Låt $x$ representera antalet pojkar som finns i varje utsaga ($0\le~x\le~10$, $x \in \mathbb{N}$).
	\[A:~x=5\]
	\[B:~x\le 4\]
	\[C:~x\ge 3\]
	\[D:~x\ge 5\]
	\[E:~x\le 8\]
	$A$ är alltså en delmängd av $C$, $D$ och $E$. $B$ är en delmängd av $E$ och $D$ är en delmängd av $C$.
	
	\ans $A \ra C,~ A \ra D,~ A \ra E,~ B \ra E,~ D \ra C$
\end{task}