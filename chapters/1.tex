%Chapter 1
\chapter{1}{Grundläggande begrepp och terminologi}

\begin{task}{1.1 a)}
	De naturliga talen ($\mathbb{N}$) innefattar alla heltal som är noll eller större. $\tfrac{6}{2}=3,~\tfrac{3}{0.1}=30,~\tfrac{0}{5}=0$.
	
	\ans $\frac{6}{2},~0,~3,~\frac{3}{0.1},~\frac{0}{5}$
\end{task}

\begin{task}{b)}
	De hela talen ($\mathbb{Z}$) inkluderar de naturliga talen ($\mathbb{N}$) samt alla negativa heltal. $-\frac{0.3}{0.02}=-15$.

	\ans $\frac{6}{2},~0,~3,~-3,~\frac{3}{0.1},~-\frac{0.3}{0.02},~\frac{0}{5}$
\end{task}

\begin{task}{c)}
	Rationella tal ($\mathbb{Q}$) är tal som kan skrivas som bråk (inkluderar de hela talen ($\mathbb{Z}$)). $3=\frac{3}{1}$ osv\ldots
	
	\ans $\frac{6}{2},~0,~3,~-3,~\frac{3}{0.1},~\frac{3}{5},~\frac{5}{3}~-\frac{0.3}{0.02},~\frac{0}{5}$
\end{task}

\begin{task}{d)}
	Reella tal ($\mathbb{R}$) är alla ``vanliga'' tal (inte de komplexa talen ($\mathbb{C}$)).
	
	\ans $\frac{6}{2},~0,~3,~-3,~\frac{3}{0.1},~\frac{3}{5},~\frac{5}{3},~\sqrt{2},~-\frac{0.3}{0.02},~\frac{0}{5},~\pi$
\end{task}

\begin{task}{1.2}
	Alla tal med ändligt antal decimaler kan skrivas som rationella tal ($1.41421=\frac{141421}{100000}$). Vi antar att ett irrationellt tal $i$ plus ett rationellt tal $r_1$ blir det rationella talet $r_2$. $i+r_1=r_2 \Rightarrow i=r_2-r_1$. Eftersom alla bråk går att skriva ihop som ett bråk stämmer inte antagandet. Svaret måste alltså bli irrationellt.
	
	\ans Nej, båda blir irrationella.
\end{task}